% file : part1.tex
% author : mxeis
% date : Sat 11 Feb 2012 09:07:53 AM UTC

\section{Definitions} % [fold]
\label{sec:Definitions}
    Tous les systemes critiques sont des systemes qui attraient aux systemes.
    Les systemes utilisent de plus en plus de generation de code.

    Un systeme: materiel, logiciel, structure d'acceuil. Un systeme est temps
    reel si l'ensemble des composants le sont.
    Surete: capacite a apporter des reponses coherentes en systemes.
    Notion de maintenabilite.
    Notion de securite a ne pas confondre avec surete (fonctionnement global).
    Securite -> au sens informatique.
    Prise en compte des couts de possesions globales des systemes.

    Comment passer d'une somme de produit a un produit de sommes.
    SP = PS.
    y = ABC|D + A|B|CD + A|BD| + AB|C + A|B|C|D
    Y|= (A| + B| + C + D)(A + B + C| + D|)(A + B| + D)(A| + B + C|)(A + B + C
    + D|)
    = A|A + A|B + A|C| + A|D| + B|A + B|B + B|C| + B|D| + ...
    \textbf{Attention} On doit toujours se ramener a une somme de produit.

    \underline{Karnault}
	On voit instantanement ce qui depend d'une valeur:
	on appelle ca le code de gray:
		AB
	|--|00|01|11|10
	|00|  |1 |  |
    C|D |01|1 |  |  |
	|11|  |  |1 |
	|10|  |1 |1 |

    Le xor  -> sert a des inversion controlees.
	    -> base meme du comparateur.
	    -> Addition.


    table de verite d'un multiplexeur 8 vers 1.

    table de verite d'un encodeur de verite a 8 entree.
    Cela sert a prioriser les interruptions, les bits de niveaux prioritarise
    les interruption, permet de les numeroter.

    --- bascules ---
    sequentiel => gestion du temps => horloge
    a chaque front montant, on recopie l'entree sur la sortie q.
    Attention au temps de propagation du systeme.
    On definit un temps de setup qui est le temps minimum ou la
    stabilite de l'entree avant le top d'horloge.
    Il faut garantir la stabilite du systeme pendant un certain temps appele
    le temps setup thole.

    Tout systeme sequentiel peut se decomposer comme etant une memoire ayant
    une fonction combinatoire permettant de definir ce dont on a besoin. C'est
    fonction combinatoire est une somme de produit.

    methode pour coder -5:  0000 0101
    complement a 1 :	    1111 1010
    complement a 2 :	    1111 1011

    un demi aditionneur : prend deux bit.

    Qu'es-ce qu'une sortie tri-step
	La gestion tri set a pour but de gerer des sorties actives a des
	sorties indifferentes, haute impedance. Cela implique la consomation /
	generation d'une importante quantite de courant.

    Qu'es-ce qu'une machine moore:
	machine a etat dont la commutation des etats est relative a la
	comutatioon des sommes des fonctions des entrees.

	Dans un systeme a etats: il faut identifier les etats, identifier les
	actions, identifier les actions.

    A quoi sert le compteur ordinal du proc?
	C'est un pointeur sur le code execute. Il sert est executer le code de
	maniere sequentielle.
	Le pointeur ordinal pointe sur toute la memoire, le pointeur
	d'instruction ne pointe que sur les instruction, l'ordinal pointe sur
	toute la memoire.

    Une micro-instruction est une instruction atomique.
	Cela garantie que cette instruction est executee en une seule fois
	sans etre interrompue. Les micro-insruction composent des instructions
	elementaire, et elles sont insecables.

    Un sequenceur : sequence -> memoriser et enchainer les etats. travail sur
    des machines a etats finit.

    Sur les FPGA, les memoire ont 16 adresses sur un bit.
    Le FPGA etant un composant a base de memoire, il est initialise par a
    l'aide d'un design stocke en memoire.

    Quelle difference entre logique positive et negative?
    logique positive -> 1 = vrai, 0 = faux.
    La logique positive peut entrainer la generation de parasites.
    logique negative -> 0 = vrai, 1 = faux.

% [end] section Definitions


\section{Questions sur les slides} % [fold]
\label{sec:Questions sur les slides}
    \subsection{Logique synchrone} % [fold]
    \label{sub:Logique synchrone}
	Le temps d'horloge > temps de hold + temps combinatoire + temps set-up
	La logique combinatoire est tjr asynchrone.
    % [end] subsection Logique synchrone

    \subsection{Logique asynchrone} % [fold]
    \label{sub:Logique asynchrone}
	Quand on est asynchrone, il y a instabilite.
	Lorsque l'on est asynchrone, la periode peut malgre etout est plus
	courte que la frequence d'horloge.
    % [end] subsection Logique asynchrone

    \subsection{La notion d'asservissement} % [fold]
    \label{sub:La notion d'asservissement}
	La notion d'asservissement vise a obtenir une stabilite du systeme.
	Peu importe le fait d'etre synchrone ou asynchrone, cela permet
	d'obtenir la stabilite.
	Cette notion est toujours utile.
	Typiquement un processeur qui veut dialoguer avec une memoire.
	\begin{itemize}
	    \item En asynchrone:\\
	    Le processeur fait une requete a un instant t.
	    A quel moment sait-il que la donnee est stable? Information de
	    stabilite donnee par un signal du peripherique (attention !=
	    interuption). Donc la memoire fourni le ACK au processeur puisque
	    il est asservi as une reponse du peripherique.
	    Des qu'on a le ACK, on traite, on enleve les temps intermediaire
	    au niveau des peripheriques.
	    Dans le systeme asynchrone, il n'y a plus de notions de
	    referentiels, plus de relation au top d'horloge.
	    \item En synchrone:\\
	    pas forcement besoin du ACK car le temps d'attente est dans le
	    processeur (connu a l'avance).
	    Dans le synchrone asservi, instants de references, la requette
	    sera emise pendant un instant de hold, la memoire la recoi au bout
	    d'un certain temps. Si on emet un ACK, le ack doit etre emis sur
	    un top d'horloge et etre stable avant le prochain top d'horloge -
	    le temps de set-up.
	\end{itemize}
    % [end] subsection La notion d'asservissement

    \subsection{Le DMA} % [fold]
    \label{sub:Le DMA}
	Le Direct Memory Access: permet a un peripherique d'echanger des
	donnees via un acces elementaire. Ex: bus de donnees, une memoire ->
	un proc -> un periph.
	Un DMA peut fonctionner en Read et en Write. Cela se fait sur le
	controle du micropros. Notion de delocalisation de l'intelligence dans
	le periph. afin de faire le transfert de maniere automatique mais sous
	le controle du proc. Le proc donnee l'adresse memoire de debut, la
	taille, l'ordre (read/write) et la memoire renvoie une interruption
	lorsque elle a terminee. Cela permet de gerer les entrees sortie de
	maniere //. Passage des donnees par petits paquets pour ne pas bloquer
	le bus.
    % [end] subsection Le DMA

% [end] section Questions sur les slides
